\documentclass{article}
\usepackage[utf8]{inputenc}
\usepackage[T1]{fontenc}

\title{Rapport Manifold Learning}
\author{Cintia Bru, Thomas Cambon, Florent Jakubowski }
\author{
  Cintia Bru\\
  \texttt{cintia.bru@univ-lyon2.fr}
  \and
  Thomas Cambon\\
  \texttt{t.cambon@univ-lyon2.fr}
  \and
  Florent Jakubowski\\
  \texttt{florent.jakubowski@univ-lyon2.fr}
}
\date{February 2021}

\usepackage{multicol}

\begin{document}
\maketitle
\tableofcontents

\begin{multicols}{2}
\section{Objectifs}
Ce projet a pour but de conduire une étude numérique de la performance des techniques vues en cours. Nous avons appliqué des méthodes et techniques étudiées sur des jeux de données artificiels et réels. Le travail s’articule sur 3 axes: simulation, estimation, comparaison.
L'objectif et de mieux appréhender l'intéret de ces différentes techniques de réduction de dimension et de comprendre leurs pertinences et leurs limites grâce à des moyens de comparaisons.
\section{Matériels et méthodes}
Dans cette nous allons expliciter les techniques que nous avons utilisés pour ce projet.
Nous avons développé notre code avec le langage de programmation R
\section{Données}
Ici nous allons décrire pourquoi nous avons choisis ces données et les décrire. 
Pour les données nous avons choisis de prendre comme données synthétique un dataset modélisant un swissroll, une helix et une sphère. Pour le jeux de données réels nous avons pris le jeu de données MNIST.
\section{Expériences numériques sur les 
données artificielles }
Cette section a pour but de détailler de notre étude afin de rendre notre travail reproductible.
Nous avons utilisé plusieurs packages R pour générer les données et employer des techniques de réduction de dimension dessus. 
\section{Comparaison des méthodes }
Pour comparer les méthodes de réduction de dimension nous avons employé les packages dimRed et coRanking. 
\section{Application sur données réelles. }
Nous avons utilisé la même suite d'opérations sur les données synthétiques que sur les données réelles. 
\end{multicols}

\end{document}
